\documentclass[a4paper,12pt,oneside]{book}

%---------
% packages
%---------
\usepackage[dvips]{graphicx}
\usepackage[T1]{fontenc}
\usepackage[utf8]{inputenc}
\usepackage[english]{babel}
\usepackage{color}
\usepackage[dvipsnames]{xcolor}
\usepackage{fancyhdr}
\usepackage{amsmath}
\usepackage{array}
\usepackage{colortbl}
\usepackage{amssymb}
\usepackage{fancybox}
\usepackage{float}
\usepackage{epic}
\usepackage{curves}
\usepackage{subfigure}
\usepackage{subeqnarray}
\usepackage{hyperref}
\usepackage{url}
\usepackage{lscape}
%\usepackage{lettrine}
%\usepackage{textcomp}
\usepackage{multicol}
\usepackage{caption}
\usepackage{wasysym}
%\usepackage{mathbx}
\usepackage{natbib}
\usepackage{ifpdf}
\usepackage{minitoc}
\usepackage{xspace}
\usepackage{doi}
\usepackage{times}
\usepackage{tikz}
\usepackage{listings}
\usepackage{makeidx}
\usepackage{booktabs}
\usepackage[explicit]{titlesec}
%\usepackage{texgraph}
\usepackage{epic,eepic}
\usepackage{rotating}
\usepackage{etex}

\lstdefinestyle{Bash}
{language=bash,
keywordstyle=\color{blue},
basicstyle=\ttfamily,
morekeywords={charles@gher13},
alsoletter={:~$},
commentstyle=\color{dkgreen},
morekeywords=[2]{charles@gher13:},
keywordstyle=[2]{\color{red}},
literate={\$}{{\textcolor{red}{\$}}}1 
         {:}{{\textcolor{red}{:}}}1
         {~}{{\textcolor{red}{\textasciitilde}}}1,
}
\lstset{
    breaklines     = true,
    frame          = single,
    rulecolor=     \color{gray},
}

%$

\lstdefinestyle{Matlab}
{language=matlab,
keywordstyle=\color{blue},
basicstyle=\ttfamily,
morekeywords={charles@gher13},
alsoletter={:~$},
commentstyle=\color{dkgreen},
morekeywords=[2]{charles@gher13:},
keywordstyle=[2]{\color{red}},
literate={\$}{{\textcolor{red}{\$}}}1 
         {:}{{\textcolor{red}{:}}}1
         {~}{{\textcolor{red}{\textasciitilde}}}1,
}
\lstset{
    breaklines     = true,
    frame          = single,
    rulecolor=     \color{gray},
}

%$

% paths + extensions for the figures
\DeclareGraphicsExtensions{.pdf,.jpg,.JPG,.png,.PNG}
\graphicspath{
{./figures/preprocessing/},{./figures/postprocessing/},{./figures/icones/},{./figures/test_cases/},
{./figures/examples/},{./figures/gallery/},{./figures/images/},{./figures/GUI/},{./figures/analysis/},
{./figures/errors/},{./figures/advection/},{./figures/papers/},{./figures/divaonweb/}
} 
 
%------------------------
%biblio style
%------------------------
\bibliographystyle{divagher}
%\bibliographystyle{plain}
% ------------------------------------------------
% NEW COLOR
\definecolor{gris}{rgb}{0.7,0.7,0.7}
\definecolor{darkgreen}{rgb}{0.14 0.73 0.21}

% ------------------------------------------------
% LENGTH DEFINITIONS
%-------------------------------------------------
\setlength{\textwidth}{16cm}
\setlength{\textheight}{24cm}
\setlength{\headheight}{50.3pt}
\setlength{\footskip}{45pt}
\setlength{\hoffset}{-1cm}
\setlength{\voffset}{-3cm}
\setlength{\unitlength}{1cm}
\setlength{\parindent}{0pt}
% ------------------------------------------------
% new commands
%-------------------------------------------------
\renewcommand{\footrulewidth}{0.4pt}
\renewcommand{\captionfont}{\it \small}

%\renewcommand{\LettrineFontHook}{\color[gray]{0.5}}

\newcommand{\example}{\underline{Example}:\,}
\newcommand{\examples}{\underline{Examples}:\,}
\newcommand{\be}{\begin{equation}}
\newcommand{\ee}{\end{equation}}
\newcommand{\beq}{\begin{eqnarray}}
\newcommand{\eeq}{\end{eqnarray}}
\newcommand{\beqn}{\begin{eqnarray*}}
\newcommand{\eeqn}{\end{eqnarray*}}

\newcommand{\montant}{\rule{0pt}{3ex}}
% ------------------------------------------------
% ABBREVIATION
%-------------------------------------------------

\newcommand{\diva}{\textsf{Diva}\xspace}
\newcommand{\matlab}{\textsf{Matlab}\xspace}
\newcommand{\plplot}{\textsf{PlPlot}}
\newcommand{\tcltk}{\textsf{Tcl/Tk}}
\newcommand{\divaversion}{4.3}
\newcommand{\gnuplot}{\textsf{gnuplot\xspace}}
\newcommand{\divawebpage}{http://modb.oce.ulg.ac.be/mediawiki/index.php/DIVA}

\newcommand{\question}{Why do I get this error?}
\newcommand{\answer}{How to solve it?}

% ------------------------------------------------
% JMB newcommands
%-------------------------------------------------

\newcommand{\nablab}{\boldsymbol{\nabla}}
\newcommand{\statmean}[1]{\left\langle #1 \right\rangle}
\newcommand{\mean}[1]{\statmean{#1}}
\newcommand{\true}[1]{{#1}^t}
\newcommand{\analyzed}[1]{{#1}^a}
\newcommand{\observation}{ \mbox{\boldmath   $ \protect\mathrm{y} $} }
\newcommand{\forecasted}[1]{{#1}^f}
\newcommand{\Hobs}{\matr{H}}
\newcommand{\errorv}{\vect{\epsilon}}
\newcommand{\errorobs}{\vect{\epsilon}^o}

\newcommand{\sing} {\rho}
\newcommand{\vnorm}[1] { \parallel \! #1 \! \parallel }
\newcommand{\taumax}[1] { \mu_{#1}^{max} }

\newcommand{\vecti}[1] { \mbox{\boldmath   $ \protect#1 $} }
\newcommand{\vects}[1] {{\boldsymbol {#1}}}
\newcommand{\vect}[1] {{\vec{#1}}}
\newcommand{\tens}[1] { \mbox{\boldmath   $ \protect\mathsf{#1} $} }
\newcommand{\matr}[1]{\mbox{$\mathbf{#1} $} }
\newcommand{\trcon}[1]{{#1}^\star}
\newcommand{\adj}[1]{\trcon{#1}}
\newcommand{\transp}[1]{{#1}^{\protect\mathrm{T}}}
\newcommand{\conj}[1]{\bar{#1}}
\newcommand{\inv}[1]{{#1}^{ \mbox{\small{-}}  1}}
\newcommand{\psinv}[1]{{#1}^{- \! \! \! 1}}
\newcommand{\noise}{\epsilon}
\newcommand{\signal}{\sigma}
\newcommand{\snr}{\lambda}

\newcommand{\ddiff}{\mbox{d}}
\newcommand{\diag}{\mbox{diag}}
\renewcommand{\vect}[1] { \mbox{\boldmath   $ #1 $} }
\renewcommand{\adj}[1]{{#1}^\mathsf{T}}
\renewcommand{\transp}[1]{{#1}^\mathsf{T}}
\renewcommand{\matr}[1] { \mbox{\boldmath   $ \protect\mathsf{#1} $} }
\renewcommand{\vect}[1] { \mbox{\boldmath   $ \protect\mathsf{#1} $} }
\newcommand{\posx}{\mathsf{r}}
\newcommand{\trace}[1]{\mathrm{trace}\left( #1 \right)}

\newcommand{\LaTeXPiX}[3]{
                          \begin{sidewaysfigure*}[ht]
                            \begin{center}
                            {\small{
                                \input{#1.eepic}
                                \caption{#2
                                \label{#3}}
                                }}
                            \end{center}
                          \end{sidewaysfigure*}
                        }
% ----------end of JMB newcommands

% ------------------------------------
% new kind of floating
%-------------------------------------
\floatstyle{boxed} 
\newfloat{exfile}{htbp}{exf}[chapter]
\floatname{exfile}{Example file}
% ------------------------------------------------

\newtheorem{tips}{Tips}[chapter]

\newcommand{\btips}{%\rule{1ex}{1ex}\,\,
\begin{tips}}
\newcommand{\etips}{\,\,\rule{1ex}{1ex}%
\end{tips}}

% level of diffuculty for the user
\newcommand{\beginer}{$\star$}
\newcommand{\intermediate}{$\star\star$}
\newcommand{\expert}{$\star\star\star$}

\newcommand{\directory}[1]{\texttt{\color{ForestGreen}{#1}}}
\newcommand{\file}[1]{\texttt{\color{MidnightBlue}{#1}}}
\newcommand{\command}[1]{\texttt{\color{RedOrange}{#1}}}

\def\BibTeX{{\rm B\kern-.05em{\sc i\kern-.025em b}\kern-.08em
    T\kern-.1667em\lower.7ex\hbox{E}\kern-.125emX}}
    
% ------------------------------------------------
% hyperref setup
%-------------------------------------------------
\hypersetup{bookmarksopen=true,
bookmarksnumbered=true,  
pdffitwindow=false, 
pdfstartview=FitP,
pdftoolbar=true,
pdfmenubar=true,
pdfwindowui=true,
pdfauthor={C.~Troupin, M.~Ouberdous, J.-M.~Beckers},
pdftitle={Diva User Guide},
pdfsubject={Geostatistics analysis tool},
bookmarksopenlevel=2,
colorlinks=true,%
breaklinks=true,%
colorlinks=true,%
linkcolor=blue,anchorcolor=blue,%
citecolor=blue,filecolor=blue,%
menucolor=black,%
urlcolor=blue}

% ------------------------------------------------
% page style
%-------------------------------------------------

\pagestyle{fancy} %Forces the page to use the fancy template
\renewcommand{\chaptermark}[1]{\markboth{\chaptername\ \MakeUppercase{\thechapter.\ #1}}{}}
\renewcommand{\sectionmark}[1]{\markright{\thesection.\ #1}}
%The text used in the header is determined by the arguments to the \markboth

\fancyhf{} %Clears all header and footer fields, in preparation.
\fancyhead[L]{\parbox{8cm}{\flushleft\textcolor{gris}{\leftmark}\\
\rule{\textwidth}{0pt}}}
\fancyhead[C]{\parbox{\textwidth}{\rule{0pt}{2.8ex}\\
\textcolor{gris}{\rule{\textwidth}{1pt}}}}
\fancyhead[R]{\parbox{6cm}{\flushright\textcolor{gris}{\rightmark}\\
\rule{\textwidth}{0pt}}} 
\renewcommand{\headrulewidth}{0pt} %Underlines the header. (Set to 0pt if not required).
\renewcommand{\footrulewidth}{0pt} %Underlines the footer. (Set to 0pt if not required)..
%\fancyfoot[C]{\large\bfseries \textcolor{gris}{--\thepage--} }

\fancyfoot[C]{\large\thepage}

% ------------------------------------------------
% chapter style
%-------------------------------------------------

\newcommand*\chapterlabel{}
\titleformat{\chapter}
  {\gdef\chapterlabel{}
   \normalfont\sffamily\Huge\bfseries\scshape}
  {\gdef\chapterlabel{\thechapter\ }}{0pt}
  {\begin{tikzpicture}[remember picture,overlay]
    \node[yshift=-2.5cm] at (current page.north west)
      {\begin{tikzpicture}[remember picture, overlay]
        \draw[fill=lightgray] (0,0) rectangle
          (\paperwidth,2.5cm);
        \node[anchor=east,xshift=.95\paperwidth,rectangle,
              rounded corners=20pt,inner sep=11pt,
              fill=black]
              {\color{white}\chapterlabel#1};
       \end{tikzpicture}
      };
   \end{tikzpicture}
  }
\titlespacing*{\chapter}{0pt}{50pt}{-60pt}
%%%%%%%%%%%%%%%%%%%%%%%%%%%%%%%%%%%%%%%%%%%%%%%%%%%%%%%%%%%%%%%

\parskip 0.25cm
\urlstyle{tt}

% ------------------------------------------------
%% Define a new 'leo' style for the package that will use a smaller font.
\makeatletter
\def\url@leostyle{%
  \@ifundefined{selectfont}{\def\UrlFont{\sf}}{\def\UrlFont{\small}}}
\makeatother
%% Now actually use the newly defined style.
\urlstyle{leo}

%-----------------------------------------------------
% For the title page
%-----------------------------------------------------

\newcommand{\HRule}[1]{\hfill \rule{0.2\linewidth}{#1}} 	% Horizontal rule

\definecolor{grey}{rgb}{0.9,0.9,0.9} 

\makeatletter							% Title
\def\printtitle{%						
    {\centering \@title\par}}
\makeatother									

\makeatletter							% Author
\def\printauthor{%					
    {\centering \large \@author}}				
\makeatother		


\title{\diva User Guide}
\author{C.~Troupin, M.~Ouberdous, D.~Sirjacobs, A.~Alvera-Azc\'{a}rate,\\ A.~Barth, M.-E. Toussaint \& J.-M.~Beckers}
\date{}

\makeindex
