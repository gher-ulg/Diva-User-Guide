% Here we can cut and paste the content of other chapters that
% won't be used in the future, but that we may want to keep
% 

%%%%%%%%%%%%%%%%%%%%%%%%
%%%%%%%%%%%%%%%%%%%%%%%%

\section{Data selection tools (Version 1.0)}
%--------------------------------------------


This is a quick guide to use scripting tools to extract data from an ascii spreadsheet file compliant with the ODV-spreadsheet format (both full and compact version). The extracted data file can then directly be used as the input file ({\tt data.dat}) for \diva.

\subsection{Installation} 
%------------------------


Place yourself in a directory of you choice, typically {\tt diva-x.y.z}
\begin{itemize}
\item
 {\tt gunzip diva-selector.tar.gz}
 \item 
 {\tt tar -xvf diva-selector.tar} 
 \end{itemize}
This creates a directory {\tt ./selector} where you should perform the data extraction, so place yourself there ({\tt cd selector}). In this directory you find scripts {\tt divaselector} and {\tt divaguessforms}.

Use {\tt divaguessforms} on one of your ODV-spreadsheet files 
({\tt divaguess\-forms myODV\-spread\-sheet\-.txt})
to create a template {\tt select.form.guess} including
guesses on delimiters ({\tt ;} or {\tt TAB}), coordinates columns, depth columns, and date columns. Use this template to create your own forms {\tt select.form} by copying the template ({\tt cp select.form.guess select.form}).
Should you not use the ISO Date format, it can be adapted in {\tt diva\-selector} (head of the file) and {\tt diva\-guessforms} (tail of the file, look for {\tt yyyy}). To help you selection the columns, {\tt diva\-guessforms} creates a file {\tt ODVcolumns} with the title and number of each column interpreted from the input file.
 
\subsection{Batch use}
%---------------------

\begin{itemize}
\item Edit {\tt select.form} and {\tt timeselect.form} to adapt to your selection criteria.
\item Execute {\tt divaselector myODVspreadsheet.txt data.dat} to extract data based on the criteria and create a {\tt data.dat} suitable for input into \diva.
\end{itemize}

You can wrap these two steps into loops by dynamically creating {\tt select.form} (\textit{e.g.} for different depth values).

Note that {\tt divaselector} creates an {\tt awk} extraction program {\tt makeselection} that you can save under another name if you want to adapt it for further use (calculating data weights depending on depth for example).




\subsection{Form definition}
%---------------------------

The selection of fields is based in ascii file {\tt select.form}. 

In this example, longitude data are found in column 4 of the spreadsheet, latitude in column 5.
Weights and Date allow the column parameter to be zero (weights are then 1 and no date test will be performed respectively).

In detail, the file defines the structure of the input file and selection criteria
\begin{itemize}
\item
The first line contains the delimiter used in the file ($\backslash${\tt t} or {\tt ;})
\item 
The second line {\bf must} contain x coordinates information

{\tt free-name  column-in-data-file  xmin xmax}. 

Data will be taken if \texttt{x} values are in the range \texttt{xmin-xmax}. If \texttt{xmin=*}, no test on \texttt{xmin} will be performed (i.e., all values lower than \texttt{xmax} are taken). Similarly, if \texttt{xmax=*}, no test on \texttt{xmax} is performed.

\item On the obligatory third line, we have a similar structure for the y coordinate
 
{\tt free-name  column-in-data-file  ymin ymax}

\item In the fourth line the variable to be analysed, is specified, with the same structure .

\end{itemize}


These four lines are mandatory.


\begin{itemize}
\item

A fifth line can contain the relative data weights

{\tt free-name  column-in-data-file  wmin wmax}

In addition, if {\tt column-in-data-file} is zero, then relative data weight are taken 1 (and the two last parameters are without effect)

\item The sixth line is related to time-selection and only two parameters are actually used
{\tt free-name column-in-data-file}

Because of the special pattern of time, range in time is not specified by min-max values
but in a separate file \file{timeselect.form}. 
If {\tt column-in-data-file} for the time-selection is zero, no selection on dates is performed and all data satisfying the other criteria will be taken. 

\item 
Any additional line of \file{select.form} after the sixth will add selection criteria with the same structure

{\tt free-name column-in-data-file minval maxval}

\end{itemize}


The addition criteria can be used to select depth ranges for example or water masses (salinity and temperature in given ranges). They can also be used to select data corresponding to quality flag values.

In the example {\tt select.form}, temperature data will be extracted west of $-10.4$ degrees longitude, with depth
of data between 20 and 30.5 meters and salinity values below 39.

The file \file{timeselect.form} has a similar form than the {\tt select.form} but is only dealing with the allowed
ranges for years, month, days, hours and minutes (the columns position is defined in \file{select.form}).
The selection is done as for \file{select.form} with {\tt *} standing for no test. 
In the example, all data from year 1960 on are taken whenever they correspond to October, November or December.

\subsection{Tricks}
%------------------

\begin{itemize}
\item If you have several ascii input files with different structures, you can create a selection (with wild cards) for each of them that leads to an identical output structure. Then you can concatenate ({\tt cat}) these output files into a singly coherent
file.

\item If you have a very large data file, instead of running \command{divaguessforms} on this file, first create a smaller one {\tt head -1000 myfile > testfile} and apply \command{divaguessforms testfile}.

\item If the spreadheet file is created by an export from ODV, quality flags generally follow the variable to which they correspond. Hence if on the fourth line (variable to analyse) of {\tt select.form} you specify column 12 for example, on line seven of {\tt select.form} you can include 

{\tt qualityflag  13   3 3}

to select all data with quality flag value 3.

\item If you just need a few data extractions and prefer a graphical interface, import the data into ODV. Then use the
{\tt Configuration -> Selection criteria}, make your selection, go into {\tt SURFACE} (or {\tt SECTION}) mode and use the {\tt Export X/Y/Z} tool that produces a \file{win1.oai} file almost ready for \diva input as \file{data.dat}. You only need to adapt the fourth column so that it includes ones, nothing or relative weights. You can achieve this for example on a command line with:
\begin{lstlisting}[style=Bash]
[charles@gher13 divastripped]$ awk '\{print \$1,\$2,\$3,1\}' ./input/win1.oai > ./input.data.dat
\end{lstlisting}
\end{itemize}

{\bf Note:} The extraction for large data sets with {\tt divaselector} can be time consuming since it is based on non-optimized {\tt awk} coding.
 
\begin{exfile}[H]
\begin{footnotesize}
\texttt{
$\backslash$ t\\
Longitude  4  -10.4  *\\
Latitude  5  *  *\\
Temperature  10  *  *\\
Weights   0   1    1\\
Date  6  \\
Depth  8  20  30.5\\
Salinity  12  *  39\\
}
\end{footnotesize}
\caption{{\tt select.form} file content.} 
\end{exfile}


\begin{exfile}[H]
\begin{footnotesize}
\texttt{
year 1960 *\\
month 10 12\\
day * *\\
hour 0 24\\
min 0 60
}
\end{footnotesize}
\caption{{\tt timeselect.form} file content.} 
\end{exfile}


