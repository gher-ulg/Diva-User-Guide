                         }
%\usepackage{geometry}
%\geometry{scale=.9, nohead}


\chapter{VADEMECUM}
\pagestyle{empty}
\thispagestyle{empty}
\vspace{-6cm}

\vspace{-3cm}
\begin{table}
\centerline{\shortstack{{\Large{DIVADEMECUM}} \\ {{}} \\ {{}}\\ {{}} \\ {{}}}}
\begin{center}
{\small{
\begin{tabular}{c|c|c}
\hline 
{\Large{Input}} & 
\shortstack{\Large{{\sf Action}} \\
{\Large{{\tt Execution}}} } & {\Large{ Output}} \\ \hline 
{\tt mycase/input/*}  & 
\shortstack{ 
{  { }  } \\
{\sf load new case} \\
{\tt divaload mycase} } & {\tt ./input/*} \\ \hline
\shortstack{
{\tt topo.dat} \\
{\tt param.par}
}
 & 
\shortstack{
{  { }  } \\
{\sf make gridded topography} \\
{\tt divatopo [-r]  } 
\\
{  { }  }
}
& 
\shortstack{
{\tt TopoInfo.dat} { } [{\tt ./input/TopoInfo.dat}] \\
{\tt topo.grd }  { } [{\tt ./input/topo.grd }]
}
\\ \hline
 \shortstack{
{\tt topo.asc} \\
{\tt topo.gebco} 
}
 & 
\shortstack{
{\sf use dbdb or gebco topography} \\
{\tt dbdb2diva [-r]  } 
\\
{\tt gebco2diva [-r]  } 
}
& 
\shortstack{
{\tt TopoInfo.dat} { } [{\tt ./input/TopoInfo.dat}] \\
{\tt topo.grd }  { } [{\tt ./input/topo.grd }]
}
 \\ \hline
 \shortstack{
{\tt TopoInfo.dat} \\
{\tt topo.grd} \\
{\tt [contour.depth] }
}
 & 
\shortstack{
{\sf make contours} \\
{\tt divacont [-r]  } 
\\
{  { }  }
}
& 
\shortstack{
{  { }  } \\
{\tt coast.cont.*}  { } [{\tt ./input/coast.cont.*}] \\
{  { }  }
}
 \\ \hline
 \shortstack{
 {  { }  } \\
{\tt param.par} \\
{\tt coast.coa} 
}
 & 
\shortstack{
{\sf use ODV contours} \\
{\tt divacoa2cont [-r]  } 
\\
{  { }  }
}
& 
\shortstack{
{  { }  } \\
{\tt coast.cont}  { } [{\tt ./input/coast.cont}] \\
{  { }  }
}
 \\ \hline
 \shortstack{
{\tt param.par} \\
{\tt coast.cont} 
}
 & 
\shortstack{
{\sf check hand-made contours} \\
{\tt divacck [-r] [-v]  } 
\\
{  { }  }
}
& 
\shortstack{
{  { }  } \\
{\tt coast.cont.checked}  { } [{\tt ./input/coast.cont}] \\
{  { }  }
}
 \\ \hline
 ../*/fort.*
 & 
 \shortstack{
 {\sf clean up directories} \\
{\tt divaclean  } 
}
& 
 { }
 \\ \hline
 \shortstack{
{\tt data.dat} \\
{\tt coast.cont} \\
{\tt [./output/fieldatdatapoint.anl]} 
}
 & 
\shortstack{
{\sf eliminate useless data} \\
{\tt divadataclean  [f$_{\min}$  f$_{\max}$]  } 
\\
{  { }  }
}
& 
\shortstack{
{  { }  } \\
{\tt ./input/data.dat} \\
{  { }  }
}
 \\ \hline
 \shortstack{
{\tt data.dat} \\
{\tt coast.cont} \\
{\tt [./output/fieldgher.anl]} 
}
 & 
\shortstack{
{\sf bins of data coverage} \\
{\tt divadatacoverage  [-n] [-r]  } 
\\
{  { }  }
}
& 
\shortstack{
{  { }  } \\
{\tt DATABINS*.dat} \\
{  {\tt RL*.dat }  }
}
 \\ \hline
 \shortstack{
 {  { }  } \\
{\tt param.par} \\
{\tt data.dat} \\
{  { }  } 
}
 & 
\shortstack{
{\sf estimate L and S/N} \\
{\tt divafit [n] [-r]  } 
\\
{  { }  } \\
{  { }  }
}
& 
\shortstack{
{\tt covariance.dat} \\
{\tt covariancefit.dat} \\
{\tt paramfit.dat} \\
{\tt param.par.fit} {  { }  } [{\tt ./input/param.par}]
}
\\ \hline
 \shortstack{
 {  { }  } \\
{\tt param.par} \\
{\tt coast.cont} \\
{ [{\tt coast.cont.dens}]  } 
}
 & 
\shortstack{
{\sf make FE mesh} \\
{\tt divamesh    } 
\\
{  { }  }
}
& 
\shortstack{
{  { }  } \\
{\tt divamesh outputs} \\
{  { }  } 
} 
\\ \hline
 \shortstack{
 {\tt gcvsampling.dat} \\
 {\tt param.par} \\
 {\tt data.dat} \\
 {\tt divamesh outputs} \\
 {[{\tt Uvel.dat,Vvel.dat}} \\
 { $\quad ${\tt UVinfo.dat, constraint.dat}]} \\
 {[{\tt RL.dat, RLinfo.dat}]} 
 }
 & 
\shortstack{
{  { }  } \\
{\sf optimise S/N by cross-validation} \\
{\tt divacv [-r]   } \\
{\tt divacvrand ns nt [-r] }  \\
{\tt divagcv [-r]   } \\
{  { }  } \\
{  { }  }
}
& 
\shortstack{
{  { }  } \\
{  { }  } \\
{\tt gcv.dat} \\
{\tt gcvsnvar.dat} \\
{\tt gcvval.dat} \\
{\tt param.par.gcv} {[{\tt ./input/param.par}] } \\
{  { }  } 
} 
\\ \hline
 \shortstack{
 {  { }  } \\
 {\tt param.par} \\
 {\tt data.dat} \\
 {\tt divamesh outputs} \\
 {[{\tt Uvel.dat,Vvel.dat}} \\
 { $\quad ${\tt UVinfo.dat, constraint.dat}]} \\
 {[{\tt RL.dat, RLinfo.dat}]} \\
 {[{\tt valatxy.coord}] }
}
 & 
\shortstack{
{  { }  } \\
{\sf make analysis} \\
{  { }  } \\
{\tt divacalc    } 
\\
{  { }  } \\
{  { }  } \\
{  { }  }
}
& 
\shortstack{
{  { }  } \\
{\tt GridInfo.dat} \\
{\tt field*.anl} \\
{\tt error*.anl} \\
{\tt *.nc} \\
{  { }  } \\
{[{\tt valatxyascii.anl}]}
} 
\\ \hline
 \shortstack{
 {  { }  } \\
 {\tt param.par} \\
 {\tt data.dat} \\
 {\tt ./output/meshvisu/*} \\
 {[{\tt Uvel.dat,Vvel.dat}} \\
 { $\quad ${\tt UVinfo.dat, constraint.dat}]} \\
 {[{\tt RL.dat, RLinfo.dat}]} 
}
 & 
\shortstack{
{  { }  } \\
{\sf perform full quality control} \\
{\tt divaqc    } 
\\
{  { }  } \\
{  { }  } \\
{  { }  }
}
& 
\shortstack{
{  { }  } \\
{  { }  } \\
{{\tt outliers.dat}} \\
{{\tt outliers.normalized.dat}} \\
{  { }  } \\
{  { }  } 
} 
\\ \hline
 \shortstack{
 {\tt param.par} \\
 {\tt data.dat} \\
{\tt divacalc outputs}
}
 & 
\shortstack{
{  { }  } \\
{\sf perform simple quality control} \\
{\tt divaqcbis } 
\\
{  { }  } \\
{  { }  } \\
{  { }  }
}
& 
\shortstack{
{  { }  } \\
{  { }  } \\
{  { }  } \\
{{\tt outliersbis.dat}} \\
{{\tt outliersbis.normalized.dat}} \\
{  { }  } \\
{  { }  } 
} 
\\ \hline
 \shortstack{
 {\tt param.par} \\
 {\tt data.dat} \\
{\tt divacalc outputs}
}
 & 
\shortstack{
{  { }  } \\
{\sf perform simple quality control} \\
{\tt divaqcter } 
\\
{  { }  } \\
{  { }  } \\
{  { }  }
}
& 
\shortstack{
{  { }  } \\
{  { }  } \\
{  { }  } \\
{{\tt outlierster.dat}} \\
{{\tt outlierster.normalized.dat}} \\
{  { }  } \\
{  { }  } 
} \\ \hline
{\tt ./output/*}  & 
\shortstack{
{\sf make some plots} \\
{\tt divagnu [f$_{\min}$ f$_{\max}$] } 
} & {\tt ./gnuwork/plots/*} \\ \hline
{\tt ./output/*}  & 
\shortstack{
{\sf save results} \\
{\tt divasave mycase} 
}
& {\tt mycase/outut/*} \\ \hline
\end{tabular}
\caption{DIVA in- and outputs. When not specified differently, input files are from directory  {\tt ./input} and 
output files are placed in directory {\tt ./output}.  Script {\tt divarefe} takes the same inputs as {\tt divacalc}
while {\tt divaanom} and {\tt divasumup} use no other user-provided files than the other scripts. Brackets {\tt [ ]} enclose optional files or parameters. Ex. {\tt [-r]} will replace an input file by the outputs from the scripts.
}
\label{vdm}
}}
\end{center}
\end{table}


\pagebreak
\clearpage
\pagebreak

\LaTeXPiX{divademecum}{Scripts used in the command-line version of DIVA; optional arguments are between {\tt []}.}{vade}

\pagebreak
\clearpage
\thispagestyle{empty}
{\Huge{Input files}}
\begin{multicols}{2}
\thispagestyle{empty}
{
\begin{minipage}{9cm}
\rule{\textwidth}{10pt}
%\begin{figure}
{\scriptsize{
\begin{verbatim}
# Correlation length (in units of data, if degrees: S-N)
1
# icoordchange (-xscale, 0=none, 1=degtokm, 2=sin projection)
0
# ispec (error output files required)
7
# ireg (subtraction of reference field 0: no, 1:mean, 2:plane)
0
# xori (origin of output regular grid, min values of x)
-4.999
# yori (origin of output regular grid, min values of y)
-4.999
# dx (x-step of output grid)
0.1999
# dy (y-step of output grid)
0.1999
# nx number of x points of output grid
51
# ny number of y points of output grid
51
# valex (exclusion value)
-9999.0
# snr signal to noise ratio 
10
# varbak variance of the background field
1
\end{verbatim}
}}
\makebox[\textwidth]{\hrulefill}
{{\tt param.par} file content. Parameters are self-explaining, except for error output specification. {\tt ispec=0} means no error field requested; add +1 for a gridded error field, +2 for error at data location and +4 for error at coordinates defined in {\tt valatxy.coord}.
From there if you want \begin{itemize}
\item error based on real covariance: {\tt ispec} $\leftarrow$ {\tt -ispec}
\item error based on real covariance with boundary effect: {\tt ispec} $\leftarrow$ {\tt -ispec-10}
\item poor man's error estimate (quick and underestimated error field): {\tt ispec} $\leftarrow$ {\tt ispec+10}
\end{itemize}
(ex: {\tt ispec=12} makes a poor mans error estimate at data locations) }
\end{minipage}
}

\begin{minipage}{9cm}
\rule{\textwidth}{10pt}
%\begin{figure}
{\scriptsize{
\begin{verbatim}
-5  22 34.8
-2  19 36.1
...
\end{verbatim}
}}
\makebox[\textwidth]{\hrulefill}
{{\tt data.dat} file content. Simple ascii file with {\tt x,y,val} and optional fourth column containing the relative weight on the data (large value = high confidence)}. {\tt topo.dat} is just a special case where the third column represents depth (positive for sea values).
\end{minipage}

\begin{minipage}{9cm}
\rule{\textwidth}{10pt}
%\begin{figure}
{\scriptsize{
\begin{verbatim}
100 10
\end{verbatim}
}}
\makebox[\textwidth]{\hrulefill}
{{\tt constraint.dat} file content. 
First value= weight on advection constraint, second value=diffusion coefficient in advection/diffusion equation.}
\end{minipage}

\begin{minipage}{9cm}
%\makebox[\textwidth]{\hrulefill}
\rule{\textwidth}{10pt}
%\begin{figure}
{\scriptsize{
\begin{verbatim}
0.01
0.03
0.1
0.3
1
3
10
30
100
\end{verbatim}
}}
\makebox[\textwidth]{\hrulefill}
{{\tt gcvsampling.dat} file content. A list of trial values for the signal/noise ratio used in cross validation tools.}
\end{minipage}

\begin{minipage}{9cm}
%\makebox[\textwidth]{\hrulefill}
\rule{\textwidth}{10pt}
%\begin{figure}
{\scriptsize{
\begin{verbatim}
2000
1500
1000
800
700
600
500
400
300
200
100
0
\end{verbatim}
}}
\makebox[\textwidth]{\hrulefill}
{{\tt contour.depth} file content with depth for contours and subsequent analysis.}
\end{minipage}

\begin{minipage}{9cm}
%\makebox[\textwidth]{\hrulefill}
\rule{\textwidth}{10pt}
%\begin{figure}
{\scriptsize{
\begin{verbatim}
0
10
0.1
0.2
101
51
\end{verbatim}
}}
\makebox[\textwidth]{\hrulefill}
{{\tt *info.dat} file desribing the gridding parameters of binary gridded files such as {\tt Uvel.dat}, {\tt topo.grd}, {\tt RL.dat}, {\tt fieldgher.anl}, {\tt errorfieldgher.anl}. Here first grid point in (0,10), with steps (0.1,0.2) and 101x51 grid points. Look at examples how to read/write binary files with Fortran or Matlab}
\end{minipage}



\end{multicols}

\pagebreak
\clearpage
{\Huge{Output files}}
\begin{multicols}{2}
\thispagestyle{empty}


\begin{minipage}{9cm}
%\makebox[\textwidth]{\hrulefill}
\rule{\textwidth}{10pt}
%\begin{figure}
{\scriptsize{
\begin{verbatim}
 0.50E+01 1131 0.16E+02 0.43E+02 0.38E+02 0.37E+02 0.13E+00
 ...
 
  flag    ident   x   y   dataval  analysis expected-misfit
\end{verbatim}
}}
\makebox[\textwidth]{\hrulefill}
{{\tt outliers*.dat} Sorted ouliers, from most suspect to least suspect. Column 1: outlier indicator (larger than 3 suspect), following columns: data identifier, x and y coordinates, original data value, analysed data value, expected misfit.}
\end{minipage}

\begin{minipage}{9cm}
%\makebox[\textwidth]{\hrulefill}
\rule{\textwidth}{10pt}
%\begin{figure}
{\scriptsize{
\begin{verbatim}
Correlation length (in degrees latitude)
  3.69890785
 Signal to noise ratio
  0.902823746
 VARBAK
  16.7839489
 For information: correlation length in km is  412.962524
\end{verbatim}
}}
\makebox[\textwidth]{\hrulefill}
{{\tt paramfit.dat} Self explaining output from {\tt divafit}. When option {\tt [-r]} is used with {\tt divafit}, an adapted {\tt param.par} will be placed in {\tt ./input}.}
\end{minipage}

\begin{minipage}{9cm}
%\makebox[\textwidth]{\hrulefill}
\rule{\textwidth}{10pt}
%\begin{figure}
{\scriptsize{
\begin{verbatim}
 S/N
  1.99764168
 VARBAK
  1.14215052
\end{verbatim}
}}
\makebox[\textwidth]{\hrulefill}
{{\tt gcvsnvar.dat} Self explaining output from {\tt divacv, divagc, divacvrand}. When option {\tt [-r]} is used with cross validation, an adapted {\tt param.par} will be placed in {\tt ./input}}.
\end{minipage}

\end{multicols}
