
\vspace*{\fill}


%\addcontentsline{toc}{chapter}{Acknowledgments}


{\it
\Large{Acknowledgments}
\vspace{1cm}
\parindent 1cm
\begin{center}
\begin{minipage}[c]{.85\textwidth}
\normalsize
This document was written for helping oceanographers work with the \diva software. This would not have been possible without the help of scientists involved in Data Analysis projects.
\newline
\newline
We would like to thank:
\newline
\newline
the participants to the \diva workshops in Li\`{e}ge (November 2006), Calvi (November 2007, October 2008, October 2009 and November 2010) and Roumaillac (October 2012) for their numerous valuable comments to improve the software and the manual;
\newline
\newline
J.~Carstensen (Aarhus University, Denmark) for his contribution in the implementation of the \textit{detrending} method;
\newline
\newline 
the National Fund for Scientific Research (FRS-FNRS, Belgium) for funding the post-doctoral positions of A.~Alvera-Azc\'{a}rate and A.~Barth, as well as supercomputer facilities;
\newline
\newline
the Fund for Research Training in Industry and Agriculture (FRIA) for funding Damien and Charles PhD grants.
\newline
Prof. Nielsen who developed the parallel skyline solver which we adapted for use with \diva \citep{NIELSEN12}.
\newline
\newline
\diva was first developed during E.U.~MODB and SeaDataNet projects; the research leading to the last developments of \diva has received funding from the European Union Seventh Framework Programme (FP7/2007-2013) under grant agreement No. 283607, SeaDataNet 2, and from project EMODNET (MARE/2008/03 - Lot 3 Chemistry - SI2.531432) from the Directorate-General for Maritime Affairs and Fisheries. 
\vspace{2cm}
\end{minipage}

\end{center}
}

\vspace*{\fill}

\newpage
