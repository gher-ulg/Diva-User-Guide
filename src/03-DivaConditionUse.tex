
\vspace*{\fill}

\begin{center}
\begin{minipage}[c]{.85\textwidth}

\Large{Conditions of use}
%\addcontentsline{toc}{chapter}{Conditions of use}
\vspace{1cm}
\normalsize

 
\vspace{.25cm}
\diva is a software developed at the GeoHydrodynamic and Environmental Research (GHER, \url{http://labos.ulg.ac.be/gher/}) group at the University of Liège (\url{https://www.uliege.be})) and further developed for SeaDataNet scientific data products in \textsf{JRA4} activities. \diva is copyright $\copyright$  2006-\the\year\xspace by the GHER group and is distributed under the terms of the GNU General Public License (GPL): \url{http://www.gnu.org/copyleft/gpl.html} \index{SeaDataNet}

In short, this means that everyone is free to use \diva and to redistribute it on a free basis. \diva is not in the public domain; it is copyrighted and there are restrictions on its distribution (see the license \url{https://www.gnu.org/copyleft/gpl.html} and its associated FAQ \url{https://www.gnu.org/licenses/gpl-faq.html}). For example, you cannot integrate this version of \diva (in full or in parts) in any \textit{closed-source} software you plan to distribute (commercially or not).

If you want to integrate \diva into a closed-source software, or want to sell a modified closed-source version of \diva, please contact us in person. You can purchase a version of \diva under a different license, with \textit{no strings attached} (for example allowing you to take parts of \diva and integrate them into your own proprietary code).

\vspace{.25cm}
All SeaDataNet~2 products (data and software) are freely distributed to the scientific community at the following conditions: 

\begin{description}
\item[\checkmark] The products should be used for scientific purposes only.
\item[\checkmark] Articles, papers, or written scientific works of any form, based in whole or in part on data or software supplied by SeaDataNet, will contain a suitable acknowledgement to the SeaDataNet~2 program of the European Union. Related publications (see bibliography) should also be cited.
\item[\checkmark] The applications of SeaDataNet~2 products are under the full responsibility of the users; neither the Commission of the European Communities nor the SeaDataNet~2 partners shall be held responsible for any consequence resulting from the use of SeaDataNet~2 products. 
\item[\checkmark] The recipient of these data will accept responsibility of informing all data users of these conditions.
\end{description}

\end{minipage}

\end{center}

\vspace*{\fill}

\newpage

\vspace*{\fill}

\begin{center}
\begin{minipage}[c]{.85\textwidth}
%\parskip 0.25cm
\Large{How to use this guide?}
%\addcontentsline{toc}{chapter}{Conditions of use}
\vspace{1cm}
\normalsize

This \diva User Guide aims to cover all the aspects of the methods: the theory (Part~\ref{part:theory}), the two-dimension version (Part~\ref{part:2Dimplementation}), the climatology production with GODIVA \index{GODIVA}(Part~\ref{part:godiva}) and the description of the scripts and the Fortran code (Part~\ref{part:appendix}). 

The user who directly wants to perform analysis shall start with Part~\ref{part:2Dimplementation}, which describes the input files and provides examples of realistic, simple runs. The \diva-demecum (Chapter~\ref{chap:divademecum}) is particularly useful to have a small summary of all the commands and options.

For more theoretical developments, the user is invited to read Part~\ref{part:theory} as well as the corresponding bibliography.

To make easier the reading of the document, different text font and colors are used for different type of files:
\begin{itemize}
\item \file{the files} (ascii or binary),
\item \command{the commands} (which can also be ascii files, but that are executable),
\item \directory{the directories}.
\end{itemize}
Various example files are provided for different situations. 


\vspace{.25cm}
\end{minipage}

\end{center}

\vspace*{\fill}

\newpage

\vspace*{\fill}

\begin{center}
\begin{minipage}[c]{.85\textwidth}
%\parskip 0.25cm
\Large{How to cite?}
%\addcontentsline{toc}{chapter}{Conditions of use}
\vspace{1cm}
\normalsize


\begin{itemize}
\item This document:


Troupin, C.; Ouberdous, M.; Sirjacobs, D.; Alvera-Azcárate, A.; Barth, A.; Toussaint, M.-E.; Watelet, S. \& Beckers, J.-M. (\the\year) \diva User Guide.\\
gher-ulg/Diva-User-Guide: v1.0 (Version v1.0). Zenodo. \doi{10.5281/zenodo.836723}


\item and the related peer-reviewed publication:


Troupin, C.; Sirjacobs, D.; Rixen, M.; Brasseur, P., Brankart, J.-M.; Barth,
A.; Alvera-Azc\'{a}rate, A.; Capet, A.; Ouberdous, M.; Lenartz, F.;
Toussaint, M.-E \& Beckers, J.-M. (2012).
Generation of analysis and consistent error fields using the Data
Interpolating Variational Analysis (Diva).
\emph{Ocean Modelling}, \textbf{52-53}: 90--101.\\
\doi{10.1016/j.ocemod.2012.05.002}\\
\url{http://www.sciencedirect.com/science/article/pii/S1463500312000790}

\item[]

{\scriptsize
\BibTeX code:
\begin{verbatim}
@ARTICLE{TROUPIN2012bOM,
  author = {C. Troupin and D. Sirjacobs and M. Rixen and 
            P. Brasseur and J.-M. Brankart and A. Barth and 
            A. Alvera-Azc\'{a}rate and A. Capet and M. Ouberdous and 
            F. Lenartz and M.-E. Toussaint and J.-M. Beckers},
  title = {{Generation of analysis and consistent error fields 
            using the Data 	Interpolating Variational Analysis (Diva)}},
  journal = om,
  year = {2012},
  volume = {52-53},
  pages = {90-101},
  doi = {10.1016/j.ocemod.2012.05.002},
  url = {http://www.sciencedirect.com/science/article/pii/S1463500312000790}
}
\end{verbatim}
}

\end{itemize}


\vspace{.25cm}
\end{minipage}

\end{center}

\vspace*{\fill}

